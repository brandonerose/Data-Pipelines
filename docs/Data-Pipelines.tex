% Options for packages loaded elsewhere
\PassOptionsToPackage{unicode}{hyperref}
\PassOptionsToPackage{hyphens}{url}
%
\documentclass[
]{book}
\usepackage{amsmath,amssymb}
\usepackage{iftex}
\ifPDFTeX
  \usepackage[T1]{fontenc}
  \usepackage[utf8]{inputenc}
  \usepackage{textcomp} % provide euro and other symbols
\else % if luatex or xetex
  \usepackage{unicode-math} % this also loads fontspec
  \defaultfontfeatures{Scale=MatchLowercase}
  \defaultfontfeatures[\rmfamily]{Ligatures=TeX,Scale=1}
\fi
\usepackage{lmodern}
\ifPDFTeX\else
  % xetex/luatex font selection
\fi
% Use upquote if available, for straight quotes in verbatim environments
\IfFileExists{upquote.sty}{\usepackage{upquote}}{}
\IfFileExists{microtype.sty}{% use microtype if available
  \usepackage[]{microtype}
  \UseMicrotypeSet[protrusion]{basicmath} % disable protrusion for tt fonts
}{}
\makeatletter
\@ifundefined{KOMAClassName}{% if non-KOMA class
  \IfFileExists{parskip.sty}{%
    \usepackage{parskip}
  }{% else
    \setlength{\parindent}{0pt}
    \setlength{\parskip}{6pt plus 2pt minus 1pt}}
}{% if KOMA class
  \KOMAoptions{parskip=half}}
\makeatother
\usepackage{xcolor}
\usepackage{color}
\usepackage{fancyvrb}
\newcommand{\VerbBar}{|}
\newcommand{\VERB}{\Verb[commandchars=\\\{\}]}
\DefineVerbatimEnvironment{Highlighting}{Verbatim}{commandchars=\\\{\}}
% Add ',fontsize=\small' for more characters per line
\usepackage{framed}
\definecolor{shadecolor}{RGB}{248,248,248}
\newenvironment{Shaded}{\begin{snugshade}}{\end{snugshade}}
\newcommand{\AlertTok}[1]{\textcolor[rgb]{0.94,0.16,0.16}{#1}}
\newcommand{\AnnotationTok}[1]{\textcolor[rgb]{0.56,0.35,0.01}{\textbf{\textit{#1}}}}
\newcommand{\AttributeTok}[1]{\textcolor[rgb]{0.13,0.29,0.53}{#1}}
\newcommand{\BaseNTok}[1]{\textcolor[rgb]{0.00,0.00,0.81}{#1}}
\newcommand{\BuiltInTok}[1]{#1}
\newcommand{\CharTok}[1]{\textcolor[rgb]{0.31,0.60,0.02}{#1}}
\newcommand{\CommentTok}[1]{\textcolor[rgb]{0.56,0.35,0.01}{\textit{#1}}}
\newcommand{\CommentVarTok}[1]{\textcolor[rgb]{0.56,0.35,0.01}{\textbf{\textit{#1}}}}
\newcommand{\ConstantTok}[1]{\textcolor[rgb]{0.56,0.35,0.01}{#1}}
\newcommand{\ControlFlowTok}[1]{\textcolor[rgb]{0.13,0.29,0.53}{\textbf{#1}}}
\newcommand{\DataTypeTok}[1]{\textcolor[rgb]{0.13,0.29,0.53}{#1}}
\newcommand{\DecValTok}[1]{\textcolor[rgb]{0.00,0.00,0.81}{#1}}
\newcommand{\DocumentationTok}[1]{\textcolor[rgb]{0.56,0.35,0.01}{\textbf{\textit{#1}}}}
\newcommand{\ErrorTok}[1]{\textcolor[rgb]{0.64,0.00,0.00}{\textbf{#1}}}
\newcommand{\ExtensionTok}[1]{#1}
\newcommand{\FloatTok}[1]{\textcolor[rgb]{0.00,0.00,0.81}{#1}}
\newcommand{\FunctionTok}[1]{\textcolor[rgb]{0.13,0.29,0.53}{\textbf{#1}}}
\newcommand{\ImportTok}[1]{#1}
\newcommand{\InformationTok}[1]{\textcolor[rgb]{0.56,0.35,0.01}{\textbf{\textit{#1}}}}
\newcommand{\KeywordTok}[1]{\textcolor[rgb]{0.13,0.29,0.53}{\textbf{#1}}}
\newcommand{\NormalTok}[1]{#1}
\newcommand{\OperatorTok}[1]{\textcolor[rgb]{0.81,0.36,0.00}{\textbf{#1}}}
\newcommand{\OtherTok}[1]{\textcolor[rgb]{0.56,0.35,0.01}{#1}}
\newcommand{\PreprocessorTok}[1]{\textcolor[rgb]{0.56,0.35,0.01}{\textit{#1}}}
\newcommand{\RegionMarkerTok}[1]{#1}
\newcommand{\SpecialCharTok}[1]{\textcolor[rgb]{0.81,0.36,0.00}{\textbf{#1}}}
\newcommand{\SpecialStringTok}[1]{\textcolor[rgb]{0.31,0.60,0.02}{#1}}
\newcommand{\StringTok}[1]{\textcolor[rgb]{0.31,0.60,0.02}{#1}}
\newcommand{\VariableTok}[1]{\textcolor[rgb]{0.00,0.00,0.00}{#1}}
\newcommand{\VerbatimStringTok}[1]{\textcolor[rgb]{0.31,0.60,0.02}{#1}}
\newcommand{\WarningTok}[1]{\textcolor[rgb]{0.56,0.35,0.01}{\textbf{\textit{#1}}}}
\usepackage{longtable,booktabs,array}
\usepackage{calc} % for calculating minipage widths
% Correct order of tables after \paragraph or \subparagraph
\usepackage{etoolbox}
\makeatletter
\patchcmd\longtable{\par}{\if@noskipsec\mbox{}\fi\par}{}{}
\makeatother
% Allow footnotes in longtable head/foot
\IfFileExists{footnotehyper.sty}{\usepackage{footnotehyper}}{\usepackage{footnote}}
\makesavenoteenv{longtable}
\usepackage{graphicx}
\makeatletter
\def\maxwidth{\ifdim\Gin@nat@width>\linewidth\linewidth\else\Gin@nat@width\fi}
\def\maxheight{\ifdim\Gin@nat@height>\textheight\textheight\else\Gin@nat@height\fi}
\makeatother
% Scale images if necessary, so that they will not overflow the page
% margins by default, and it is still possible to overwrite the defaults
% using explicit options in \includegraphics[width, height, ...]{}
\setkeys{Gin}{width=\maxwidth,height=\maxheight,keepaspectratio}
% Set default figure placement to htbp
\makeatletter
\def\fps@figure{htbp}
\makeatother
\setlength{\emergencystretch}{3em} % prevent overfull lines
\providecommand{\tightlist}{%
  \setlength{\itemsep}{0pt}\setlength{\parskip}{0pt}}
\setcounter{secnumdepth}{5}
\usepackage{booktabs}
\ifLuaTeX
  \usepackage{selnolig}  % disable illegal ligatures
\fi
\usepackage[]{natbib}
\bibliographystyle{plainnat}
\usepackage{bookmark}
\IfFileExists{xurl.sty}{\usepackage{xurl}}{} % add URL line breaks if available
\urlstyle{same}
\hypersetup{
  pdftitle={Data Pipelines: Using a Model-View-Controller Framework in R},
  pdfauthor={Brandon Rose and Natalie Goulett},
  hidelinks,
  pdfcreator={LaTeX via pandoc}}

\title{Data Pipelines: Using a Model-View-Controller Framework in R}
\author{Brandon Rose and Natalie Goulett}
\date{12/2024}

\begin{document}
\maketitle

{
\setcounter{tocdepth}{1}
\tableofcontents
}
\chapter{\texorpdfstring{What is \texttt{\{RosyREDCap\}}?}{What is \{RosyREDCap\}?}}\label{what-is-rosyredcap}

{[}\texttt{\{RosyREDCap\}}{]}

R and REDCap are both widely utilized in medicine, including basic science, clinical research, and clinal trials.
Both have independent strengths, but together they can create powerful data pipelines.
While several R packages exist for extracting data using the REDCap API, \texttt{\{RosyREDCap\}} stands out by offering comprehensive extraction of all metadata and data from any REDCap project into a standardized R list object, facilitating effortless downstream use in exports, imports, transformation, and applications.
Three core aims of \texttt{\{RosyREDCap\}} are to

\begin{enumerate}
\def\labelenumi{\arabic{enumi}.}
\tightlist
\item
  Maintain a local version of the database (DB) object by only calling recently updated records using the REDCap log.
\item
  Allow imports of non-coded versions of the dataset using R or Excel/CSV.
\item
  Launch a shiny app that allows you to explore all of your REDCap projects.
\end{enumerate}

By leveraging the combined strengths of R and REDCap, users can maintain strong clinical data pipelines, collected and processed appropriately to improve research and patient care.
RosyREDCap can be used as a base data object and data quality tool for most REDCap projects to aid in collection, monitoring, transformation, and analysis.

\section{Installing RosyREDCap}\label{installing-rosyredcap}

\emph{Note: The current version of \texttt{\{RosyREDCap\}} used when writing this book is 1.0.0.9030, and some of the features presented in this book might not be available if you are using an older version, or be a little bit different if you have a newer version. Feel free to browse the package NEWS.}

The stable release can be found on CRAN and installed with:

\begin{Shaded}
\begin{Highlighting}[]
\FunctionTok{install.packages}\NormalTok{(}\StringTok{"RosyREDCap"}\NormalTok{)}
\end{Highlighting}
\end{Shaded}

You can install the development version of RosyREDCap from GitHub by using the \texttt{\{remotes\}} package.
Be sure to install \texttt{\{remotes\}} if you don't have it already.

\begin{Shaded}
\begin{Highlighting}[]
\NormalTok{remotes}\SpecialCharTok{::}\FunctionTok{install\_github}\NormalTok{(}\StringTok{"brandonerose/RosyREDCap"}\NormalTok{)}
\end{Highlighting}
\end{Shaded}

Note that the version used at the time of writing this book is 1.0.0.9030.
You can check what version you have installed with the following.

\begin{Shaded}
\begin{Highlighting}[]
\FunctionTok{packageVersion}\NormalTok{(}\StringTok{"RosyREDCap"}\NormalTok{)}
\CommentTok{\#\textgreater{} [1] \textquotesingle{}1.0.0.9030\textquotesingle{}}
\end{Highlighting}
\end{Shaded}

The motivation behind \texttt{\{RosyREDCap\}} is that building a proof-of-concept application is easy, but \textbf{things change when the application becomes larger and more complex, and especially when you need to send that app to production}.
Until recently there has not been any real framework for building and deploying production-grade \texttt{\{shiny\}} apps.
This is where \texttt{\{RosyREDCap\}} comes into play: \textbf{offering \texttt{\{shiny\}} developers a toolkit for making a stable, easy-to-maintain, and robust production web application with R}.
\texttt{\{RosyREDCap\}} has been developed to abstract away the most common engineering tasks (for example, module creation, addition and linking of an external CSS or JavaScript file, etc.), so you can focus on what matters: building the application.
Once your application is ready to be deployed, \texttt{\{RosyREDCap\}} guides you through testing and brings tools for deploying to common platforms.

If you have any issues, try downloading the most recent version of R at RStudtio and update all packages in RStudio.
See \href{https://www.thecodingdocs.com/r/getting-started}{thecodingdocs.com/r/getting-started}.

\chapter{Intended Uses}\label{intended-uses}

RosyREDCap is for all three types of R users below (basic, intermediate, and advanced). For each type there is an \emph{intended} way of using the package, which may evolve over time.

\section*{The Basic R User}\label{the-basic-r-user}
\addcontentsline{toc}{section}{The Basic R User}

A basic R user is capable of installing R and RStudio on their computer. They may be very hesitant working through any errors or understanding ``how it all works''. However, they know how run some lines of code if the process is straight forward.

\section*{The Intermediate R User}\label{the-intermediate-r-user}
\addcontentsline{toc}{section}{The Intermediate R User}

An intermediate user knows how to create an R project and even has some scripts they have written themselves somewhere on their computer. The haven't gone down many rabbit holes but they know to make some ggplots and are capable of working through most common errors.

\section*{The Advanced R User}\label{the-advanced-r-user}
\addcontentsline{toc}{section}{The Advanced R User}

An advanced R user understands how R packages work. They can write reusable functions and do complex tasks that may include API calls, data transformations, working with SQL and more.

  \bibliography{book.bib,packages.bib}

\end{document}
